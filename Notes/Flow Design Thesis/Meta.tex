% Meta-Informationen -----------------------------------------------------------
%   Definition von globalen Parametern, die im gesamten Dokument verwendet
%   werden k�nnen (z.B auf dem Deckblatt etc.).
%
%   ACHTUNG: Wenn die Texte Umlaute oder ein Esszet enthalten, muss der folgende
%            Befehl bereits an dieser Stelle aktiviert werden:
%            \usepackage[latin1]{inputenc}
% ------------------------------------------------------------------------------
\newcommand{\titel}{Flow Design}
\newcommand{\untertitel}{	Konzeption und Implementierung einer WPF Anwendung zur grafischen Modellierung und	Code-basierter Generierung von Flow Design Entw�rfen unter Verwendung von C\# und Microsoft Roslyn}
\newcommand{\art}{Fakult�t 1}
\newcommand{\fachgebiet}{Informatik}
\newcommand{\autor}{Dennis M�ller}
\newcommand{\autorEmail}{dennis.briefkasten@gmail.com}
\newcommand{\studienbereich}{Medieninformatik}
\newcommand{\matrikelnr}{25675}
\newcommand{\erstgutachter}{Prof. Walter Kriha }
\newcommand{\zweitgutachter}{Kevin Erath}
\newcommand{\jahr}{2017}
\newcommand{\ort}{Esslingen am Neckar}
\newcommand{\logo}{HdM-Logo.png}


%K�rzel f�r h�ufige Fachbegriffe 
%\newcommand{\SLAM}{Simultaneous Localization and Mapping }
%\newcommand{\ROS}{Robot Operating System }
%\newcommand{\AI}{Artificial Intelligence for Robotics }


%\newcommand{\RN}{\gls{real number}} 
%\newcommand{\pose}{\gls{Pose} }

%Nummerierung der parts von Roman auf Alpha setzten
\renewcommand\thepart{\Alph{part}}
\renewcommand\thepart{\Alph{part}}