% Eigene Befehle und typographische Auszeichnungen f�r diese

% einfaches Wechseln der Schrift, z.B.: \changefont{cmss}{sbc}{n}
	\newcommand{\changefont}[3]{\fontfamily{#1} \fontseries{#2} \fontshape{#3} \selectfont}

% Abk�rzungen mit korrektem Leerraum 
	\newcommand{\ua}{\mbox{u.\,a.\ }}
	\newcommand{\zB}{\mbox{z.\,B.\ }}
	\newcommand{\dahe}{\mbox{d.\,h.\ }}
	\newcommand{\Vgl}{Vgl.}
	\newcommand{\bzw}{bzw.\ }
	\newcommand{\evtl}{evtl.\ }	
	\newcommand{\abbildung}[1]{Abbildung~\ref{fig:#1}}	
	\newcommand{\bs}{$\backslash$}

% erzeugt ein Listenelement mit fetter �berschrift 
	\newcommand{\itemd}[2]{\item{\textbf{#1}}\\{#2}}

% einige Befehle zum Zitieren --------------------------------------------------
	\newcommand{\Zitat}[2][\empty]{\ifthenelse{\equal{#1}{\empty}}{\cite{#2}}{\cite[][#1]{#2}}}
%\newcommandtwoopt{\Zitats}[2][\empty][Def5]{\ifthenelse{\equal{#1}{\empty}}{\citep{#2}}{\citep[#1][#5]{#2}}}

%Text fett schreiben 
	\newcommand{\F}[1]{\textbf{#1}}
%Mathematische Symbole Fett schreiben
	\newcommand{\FM}[1]{\boldsymbol{#1}}


% zum Ausgeben von Autoren
	\newcommand{\AutorName}[1]{\textsc{#1}}
	\newcommand{\Autor}[1]{\AutorName{\citeauthor{#1}}}

% verschiedene Befehle um W�rter semantisch auszuzeichnen ----------------------
	\newcommand{\NeuerBegriff}[1]{\textbf{#1}}
	\newcommand{\Fachbegriff}[1]{\textit{#1}}
	
	\newcommand{\Eingabe}[1]{\texttt{#1}}
	\newcommand{\Code}[1]{\texttt{#1}}
	\newcommand{\Datei}[1]{\texttt{#1}}
	
	\newcommand{\Datentyp}[1]{\textsf{#1}}
	\newcommand{\XMLElement}[1]{\textsf{#1}}
	\newcommand{\Webservice}[1]{\textsf{#1}}

% Variablen mit Text als Index mit mbox erzeugen (Viel gebastel... keine garantie)
	%\newcommand{\ind}[1]{\mbox{{\textit{#1}}}}
	\newcommand{\ind}[1]{{\textit{#1}}}

	\newenvironment{rcase}{\left\rbrace\begin{aligned}}{\end{aligned}\right.}
	\makeatletter
	\def\env@casess{%
	  \let\@ifnextchar\new@ifnextchar
	  \left.%
	  \def\arraystretch{1.2}%
	  \array{@{}l@{\quad}l@{}}%
	}
	\newenvironment{rcases}{%
	  \matrix@check%
	  \rcases%
	  \env@casess
	}{%
	  \endarray\right\}%
	}
	\makeatother
	
%Bibeinrag Autoren fett
	\AtBeginBibliography{%
	  \renewcommand*\mkbibnamefirst[1]{\bfseries{#1}}
	  \renewcommand*\mkbibnamelast[1]{\bfseries{#1}}
	  \renewcommand*\mkbibnameprefix[1]{\bfseries{#1}}
	  \renewcommand*\mkbibnameaffix[1]{\bfseries{#1}}
	  \DeclareFieldFormat{year}{\bfseries{#1}}
	  \DeclareFieldFormat{labelyear}{\bfseries{\mkbibemph{\mknumalph{#1}}}}
	}