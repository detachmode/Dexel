% Created 2016-11-02 Mi 14:07
\documentclass[11pt]{article}
\usepackage[utf8]{inputenc}
\usepackage[T1]{fontenc}
\usepackage{fixltx2e}
\usepackage{graphicx}
\usepackage{grffile}
\usepackage{longtable}
\usepackage{wrapfig}
\usepackage{rotating}
\usepackage[normalem]{ulem}
\usepackage{amsmath}
\usepackage{textcomp}
\usepackage{amssymb}
\usepackage{capt-of}
\usepackage{hyperref}
\date{\today}
\title{}
\hypersetup{
 pdfauthor={},
 pdftitle={},
 pdfkeywords={},
 pdfsubject={},
 pdfcreator={Emacs 25.1.1 (Org mode 8.3.5)}, 
 pdflang={English}}
\begin{document}

\section{Motivation}
\label{sec:orgheadline8}
Zur Zeit dieser Arbeit exisiteren keine ausgereiften Tools oder Hilfsprogrmme die speziell auf die Erstellung
von Flow Design Entwürfen ausgelegt sind. Dies liegt möglicherweise an der
geringen Bekanntheit und Verbreitung der Methodik.
Ziel dieser Arbeit ist es erstmal dem Leser anschaulich zu erklären, um was
es sich bei Flow Design im Detail handelt, sowie die dazugehörenden Implemtierungsregeln anhand von
Codebeispielen in C\# zu zeigen.
Hat man einmal den Vorteil von Flow Design für sich entdeckt liegt es nahe als
Programmierer sich Gedanken darüber zu machen wie ein Computerprogramm aussehen
könnte, das einem bei der Verwendung der Methodik so
gut es geht unterstützt. Da es solch ein Programm noch nicht gibt, geht es in
dieser Arbeit einmal darum die Ansprüche eines solchen Programms im Detail
herauszufinden, sowie nützliche Features aufzuzlisten. Ein weiterer großer Teil dieser
Arbeit besteht darin einen Grundstein für solch ein Programm in Form eines
ersten Prototypen zu legen und diesen auf Github als Open Source Projekt zu veröffentlichen.
Dieser Protopype soll in C\#  und in Teilen unter Verwendung von Flow Design selbst umgestetzt werden. 
Am Ende soll die interessanten Teile des Codes hier dokumentiert und das Ergebnis vorgestellt werden.
Endziel ist es, diesen Prototypen soweit zu bringen, dass für den Anwender einen klaren
Vorteil spürbar ist in der Verwendung des Editors im Vergleich zu Stift und Papier
, sowie die Produktivität und Komfort von der Verwendung von Flow Design in einem
Projekt steigert wird. Unabhänig davon, ob diese Ziel erreicht wurde oder nicht sollen die gewonnen
Erkenntnisse aus dem Versuch hier dokumentiert und ein Fazit daraus gezogen werden.

\subsection{Der Editor}
\label{sec:orgheadline1}
Im Grunde geht es bei solch einem Editors vorallem darum die Vorteile aus der digitalen Welt mit
der Methodik zu vereinen, ohne die Einfachheit der Methodik auf dem Papier zu
verlieren. Im folgendem einige Kerngedanken über die Funktionalität des Editors:

\subsection{Vorteile eines digitalen Editors}
\label{sec:orgheadline6}
Flow Design ist eigentlich als Entwurfsmethode auf dem Papier gedacht.
Jedoch hat ein Flow Design auf dem Papier einige Nachteile, die ein Editor am
Computer aufheben könnte. Das wären vorallem folgende Punkte:
\subsubsection{einmal eingezeichnete Pfeile, Kreise und Beschriftungen lassen sich nicht so leicht mehr verändern.}
\label{sec:orgheadline2}
Während des kreativen Prozesses ein Programmierproblem zu lösen, bedarf es
mehrer Iterationen und Veränderungen an dem Diagramm. Eine Möglichkeit Versionen
abzuspeichern und Teile zu verschieben, umzubenennen und Umzustukturieren sind
klare Vorteile von einem digitalen Editors.
\subsubsection{Unnötige Abtipparbeit ersparen.}
\label{sec:orgheadline3}
Oft ist es möglich aus den Beschriftungen die Variablennamen und
Methoden Signaturen abzuleiten. Liegt das Diagramm in digitaler Form vor, wäre eine automatische Generierung von Quellcode naheliegend und
würde einem zusätzliche Abtipparbeit ersparen.
\subsubsection{Generierung von Methoden-Bodies}
\label{sec:orgheadline4}
Durch einige Implementierungsregeln von Flow Design lassen sich von einer
bestimmten Art von Methoden sogar die komplette Implementierung aus dem Diagramm
ableiten. Das wäre ein zustätzlicher Komfortgewinn von einem Editor.
\subsubsection{Roundtrip-Engineering}
\label{sec:orgheadline5}
Eine Möglichkeit aus einer bestehenden Codebasis ein Diagramm zu erstellen -
wenn auch nur teilweise - würden die Produktivität beim Einsetzen von Flow
Design in einem Projekt weiter steigern.

\subsection{Vorteile von einem Entwurf auf dem Papier}
\label{sec:orgheadline7}
Ein Papier schränkt einen nicht ein und erlaubt es schnell und einfach Pfeile
und Kreise zu zeichnen, Notizen einzufügen und ist einfach in der Bedienung.
Bei dem Erstellen eine Editors muss deshalb ein besonders großes Augenmerk auf
eine gute und intuitive Bedienung gelegt werden, damit einem das Programm bei der kreativen Arbeit nicht
behindert und für den Anwender ein klaren Vorteil in der Verwendung des Editors
erkennt im Vergleich zum Zeichnen auf dem Papier.
\end{document}
