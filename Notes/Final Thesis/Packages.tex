\usepackage{url}

\usepackage[
backend=biber,
style=alphabetic,
]{biblatex}

\usepackage{biblatex}
\addbibresource{ref.bib}

\usepackage{array}

\usepackage{listings}


\usepackage[utf8]{inputenc}
\usepackage{color}
\usepackage{textcomp}
\definecolor{bluekeywords}{rgb}{0,0,1}
\definecolor{greencomments}{rgb}{0,0.5,0}
\definecolor{redstrings}{rgb}{0.64,0.08,0.08}
\definecolor{xmlcomments}{rgb}{0.5,0.5,0.5}
\definecolor{types}{rgb}{0.17,0.57,0.68}
\usepackage{bookmark,hyperref}
\definecolor{lightgrey}{rgb}{0.95,0.95,0.95}
\lstset{language=[Sharp]C,
	captionpos=b,
	basicstyle=\fontfamily{consolas}\selectfont
	framextopmargin=26pt,
	tabsize=3,
	framexbottommargin=0pt, 
	frame=tb, framerule=0pt,
	showspaces=false,
	showtabs=false,
	breaklines=true,
	showstringspaces=false,
	breakatwhitespace=true,
	frame=single,
	backgroundcolor=\color{lightgrey},
	escapeinside={(*@}{@*)},
	commentstyle=\color{greencomments},
	morekeywords={partial, var, value, get, set, Action, Func, IEnumerable},
	keywordstyle=\color{bluekeywords},
	stringstyle=\color{redstrings},
	basicstyle=\ttfamily\small,
}
\usepackage[ngerman]{babel}
\usepackage{tabularx}
\usepackage{fixltx2e}
\usepackage{graphicx}
\usepackage{grffile}
\usepackage{longtable}
\usepackage{wrapfig}
\usepackage{rotating}


\usepackage[normalem]{ulem}
\usepackage{amsmath}
\usepackage{textcomp}
\usepackage{amssymb}
\usepackage{capt-of}
\usepackage{hyperref}
\usepackage[autostyle=true,german=quotes]{csquotes}
\let\svthefootnote\thefootnote
\newcommand\blankfootnote[1]{%
	\let\thefootnote\relax\footnotetext{#1}%
	\let\thefootnote\svthefootnote%
}

\date{\today}
\title{Flow Design}
\hypersetup{
	pdfauthor={},
	pdftitle={Flow Design},
	pdfkeywords={},
	pdfsubject={},
	pdfcreator={Emacs 25.1.1 (Org mode 8.3.5)}, 
	pdflang={German}}


\usepackage[T1]{fontenc}

\usepackage[german, refpage]{nomencl}
\makenomenclature
\let\abk\nomenclature 


\usepackage{epigraph}
\usepackage{float}
\usepackage{tablefootnote}

% Tabellen Padding
\setlength{\tabcolsep}{10pt}
\renewcommand{\arraystretch}{1.7}

\newcommand{\titledate}[2][2.5in]{%
	\noindent%
	\begin{tabular}{@{}p{#1}@{}}
		\\ \hline \\[-.75\normalbaselineskip]
		#2
	\end{tabular} \hspace{1in}
	\begin{tabular}{@{}p{#1}@{}}
		\\ \hline \\[-.75\normalbaselineskip]
		Unterschrift
	\end{tabular}
}
