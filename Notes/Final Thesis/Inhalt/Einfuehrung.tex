
\chapter{Einführung}

\section{Motivation}

Software wird häufig einfach nur herunter-programmiert. Dadurch entsteht
umgangssprachlich bezeichnet Spaghetticode mit vielen Abhängigkeiten innerhalb
der eigenen Codebasis. Solch eine Codebasis ist schwer zu warten und auf
Änderungen anzupassen. Die Alternative dazu wäre, vor dem Programmieren einen
Entwurf der Architektur zu erstellen und sich vor der Umsetzung Gedanken zu
der Struktur zu machen.

Viele der vorhanden Entwurfsmethodiken sind jedoch zu schwergewichtig und helfen
einem nicht zwingend dabei Abhängigkeiten zu reduzieren.
Zusätzlich zu Entwurfsmethodiken existieren auch Clean Code Prinzipien, die
einem Helfen sollen, besser wartbare Software zu schreiben. Viele der
Erkenntnisse aus den Prinzipien sind jedoch leider oft nur schwer auf die eigene
Codebasis anzuwenden, da sie zu abstrakt sind.
Dadurch finden sie in der Praxis seltener Anwendung, als nötig wäre, um sauberen
Code zu schreiben.

Hier möchte Flow Design Abhilfe schaffen, indem sie eine leichtgewichtige
Entwurfsmethodik bietet, mit einem speziellen Augenmerk darauf Abhängigkeiten
zu reduzieren. Nebenbei hilft Flow Design einem auch gängige Clean Code
Prinzipien einzuhalten. 

Leider ist diese Methodik nicht weit verbreitet und das
Entwerfen ist bisher auf dem Papier angedacht. Aus diesem Grund existieren keine
ausgereiften Tools oder Hilfsprogramme die speziell auf die Erstellung
von Flow Design Entwürfen ausgelegt sind. 

Ziel dieser Arbeit ist es eine Anwendung zur Erstellung von Flow Design Diagrammen 
zu entwerfen und einen Prototypen zu implementieren. Diese Anwendung soll darüber hinaus auch
Funktionen zur Generierung von Quellcode aus den erstellten Diagrammen bieten, damit
der Einsatz von Flow Design in einem Projekt komfortabler und produktiver wird.

Unabhängig davon, ob dieses Ziel erreicht wurde oder nicht sollen die gewonnen
Erkenntnisse aus dem Versuch hier dokumentiert und ein Fazit daraus gezogen werden.

Der Prototyp soll in C\# und in Teilen unter Verwendung von Flow Design selbst umgesetzt und auf
Github als  Open-Source-Projekt veröffentlicht werden.



\section{Aufbau}

Der erste Teil dieser Arbeit stellt ein Grundlagenkapitel dar, dass dem Leser die Methodik Flow Design näher bringen soll.
Der Leser bekommt die Entwurfsmethodik Flow Design anschaulich erklärt, dabei wird auch auf die Herkunft,
den Grundgedanken dieser Methodik, sowie ihre Vor- und Nachteile eingegangen.
Dazu gehören auch, dass einige neue Prinzipien und Abkürzungen erklärt werden.
Des weiteren wird auf die für Flow Design speziellen, dazugehörigen Implemtierungsregeln eingegangen und
anhand von einfachen Codebeispielen in C\# dem Leser näher gebracht. 
Bei dieser Gelegenheit werden in diesem Teil auch auf einige Sprachfeatures 
von C\# eingegangen, die einem bei der Umsetzung von Datenströmen hilfreich sein
können. Auch hier werden einfache Codebeispiel zum besseren Verständnis dem Leser 
präsentiert. Im genauen handelt es sich hierbei um Lambdas und die Methodenbibliothek LINQ.

Hat man einmal den Vorteil von Flow Design für sich entdeckt liegt es nahe als
Programmierer sich Gedanken darüber zu machen wie eine Anwendung aussehen
könnte, das einem bei der Verwendung der Methodik so gut es geht unterstützt.
Da es solch ein Programm noch nicht gibt, geht es in diesem Teil der Arbeit
darum die Ansprüche eines solchen Programms im Detail herauszufinden und sie
in Form von Anforderungen aufzulisten. Anschließend werden GUI-Skizzen
und einige Gedanken zu der Usability der Anwendung vorgestellt. 

Aus Ende sollen ausgewählte Teile des Codes hier dokumentiert und das Ergebnis vorgestellt werden.
Außerdem wird ein Ausblick auf die Zukunft des Projektes gegeben und ein Fazit gezogen werden.

