
\chapter*{Abstract}

Der erste Teil dieser Arbeit vermittelt dem Leser ein Grundwissen über Flow Design. Eine Entwurfsmethodik die Softwareentwickler helfen soll saubereren Code zu schreiben.

Der Hauptfokus wurde auf das Flow Design-Diagramm  gelegt,
eine auf Datenfluss fokussierte Entwurfsmethodik.
Die Notation wird erläutert und die Implementierung des Datenflusses in C\#  anhand von Beispielen dem Leser nähergebracht.
Es werden Regeln und Prinzipien vorgestellt, die bei der Umsetzung des Datenflusses einzuhalten sind, um Abhängigkeiten im Code zu reduzieren.

Der zweite Teil dieser Arbeit beschäftigt sich mit der Erstellung eines Prototypen eines
Editors für Flow Design. Ein Editor mit dem es möglich ist Datenflüsse in Form von Flow Design zu modellieren. Dieser Prototyp bietet darüber hinaus die Möglichkeit C\#-Code aus diesen Flussdiagrammen zu generieren. 
Zuerst wird eine Systemanalyse durchgeführt mit einem
speziellen Fokus auf eine gute Usability.
Danach wird der aktuelle Stand des Prototypen vorgestellt, sowie ein Einblick in die Architektur gewährt. Dabei werden auch einige
ausgewählte Codeauschnitte präsentiert, die ebenfalls nach den Regeln und Prinzipien implementiert wurden, die im Grundlagenkapitel vorgestellt wurden.
Diese Codeauschnitte sind komplexer als die Beispiele aus dem Grundlagenkapitel und sollen dem Leser  einen Einblick gewähren, wie ein Code nach Flow Design in der Praxis aussehen kann.
Damit wird dem Leser eine Möglichkeit geboten, Flow Design besser für sich bewerten zu können.

Als Arbeitstitel für den Prototypen wurde der Projektname \enquote{Dexel} gewählt.

Dexel ist auf Github als Open-Source-Projekt unter folgender URL einsehbar:

\url{https://github.com/detachmode/Dexel}

Der Zustand von Dexel zum Zeitpunkt dieser Arbeit ist eine
Anwendung, die für den Einsatz in einem Projekt noch nicht ausgereift genug ist. 
Es wurde jedoch ein Grundstein gelegt, auf dem aufgebaut werden kann.
Die Anwendung hat einen Punkt erreicht hat, an dem die
Richtung des Projektes deutlich wird und eine Vision wie ein Editor für Flow Design aussehen kann, vermittelt wird.




