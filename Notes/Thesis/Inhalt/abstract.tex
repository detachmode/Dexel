
\chapter*{Abstract}

Dieses Arbeit wurde in zwei Teile aufgeteilt: 
Ein Grundlagenteil und ein Realisierungsteil.

Der Grundlagenteil vermittelt dem Leser ein Grundwissen über Flow Design.
Der Hauptfokus dabei wurde auf das Flussdiagramm gelegt.
Die Notation wird erläutert und die Implementierung des Datenflusses in C\#
wird anhand von Beispielen dem Leser nähergebracht.

Dem Leser wird im ersten Teil die Grundlagen nähergebracht, die er braucht, um die
nachfolgenden Kapitel in der Realisierung zu verstehen.

Im zweiten Teil dieser Arbeit wird die Vision erläutert und die Überlegungen wie ein
Editor für Flow Design aussehen könnte beschrieben.

Im darauf folgendem Kapitel wird der Prototyp vorgestellt, wie der akutelle Stand des
Editors ist. Dabei wurden ein großteil der Hauptfeatures umgesetzt.
Auch wurde Wert darauf gelegt, ein saubere Codebasis zu haben, die es ermöglicht 
das Projekt weiter zu entwickeln.
Der Architektur des Projektes wird vorgestellt und erklärt, warum man sich für 
diese Aufteilung entschieden hat.

Es fehlen jedoch noch ein paar Features um den Editor für Projekte einzusetzen.
Es wurde jedoch ein Grundstein gelegt, auf dem aufgebaut werden kann.
Auch nicht jede Notation wurde umgesetzt, sodass zum Beispiel Rekursionen
nicht sauber dargestellt werden können.


Es wurde ein Prototyp erstellt, um Flow Design Diagramme zu erstellen.
Auch die Generierung von C\# Code anhand der erstellten Diagramme wurde umgesetzt.
Dabei beinhaltet die Generierung sowohl die Generierung von Operationen-Signaturen und 
die komplette Generieung von Integrationen.



Der Prototyp ist auf Github als Open-Source-Projekt unter folgender URL einsehbar:

\url{https://github.com/detachmode/Dexel}



